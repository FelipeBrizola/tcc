\chapter{\label{chap:chap2} Fundamentação Teórica}

Este capítulo apresentará uma breve descrição de alguns trabalhos relacionados a área de computação em névoa, bem como protocolos de comunicação que servirão de apoio para este TCC.


\section{Trabalhos relacionados}


O protocolo BGP está situado na quinta camada, a camada de aplicação, do modelo de referência TCP/IP\cite{tanenbaum2011redes}.
Abaixo, de forma sucinta, elencaremos algumas funcionalidades básicas do protocolo que embasarão o restante deste trabalho.

\begin{itemize}
    \item A responsabilidade deste protocolo é manter a troca de informações sobre roteamentos entre sistemas autônomos\cite{Rekhter:1995}.
    \item O roteador ao entrar na rede pela primeira vez deve-se conectar ao seu vizinho. Após a conexão estabelecida, os roteadores compartilham entre sí suas tabelas de roteamento\cite{Rekhter:1995}.
    \item Posteriormente, as atualizações nas tabelas dos roteadores dão-se de forma incremental à medida que as mudanças na rotas ocorrem\cite{Rekhter:1995}.
    \item Mensagens de keep alive são trocadas periodicamente a fim de garantir conectividade entre os roteadores\cite{Rekhter:1995}.
\end{itemize}




