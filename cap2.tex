\chapter{\label{chap:chap2} Fundamentação Teórica}

Este capítulo apresentará uma breve descrição de alguns trabalhos relacionados a área de computação em névoa, bem como seus protocolos de comunicação.


\section{Trabalhos relacionados}


O protocolo BGP está situado na quinta camada, a camada de aplicação, do modelo de referência TCP/IP\cite{tanenbaum2011redes}.
Abaixo, de forma sucinta, elencaremos algumas funcionalidades básicas do protocolo que embasarão o restante deste trabalho.

A responsabilidade deste protocolo é manter a troca de informações sobre roteamentos entre sistemas autônomos\cite{rfc1163}.
De forma sucinta, a troca de informações ocorre da seguinte forma: 

\begin{itemize}
    \item O roteador ao entrar na rede pela primeira vez deve-se conectar ao seu vizinho. Após a conexão estabelecida, os roteadores compartilham entre sí suas tabelas de roteamento\cite{rfc1163}.
    \item Posteriormente, as atualizações nas tabelas dos roteadores dão-se de forma incremental à medida que as mudanças na rotas ocorrem\cite{rfc1163}.
    \item Mensagens de keep alive são trocadas periodicamente a fim de garantir conectividade entre os roteadores\cite{rfc1163}.
\end{itemize}








% Fog protocol nao é distribuido. assim as mensagens trocadas entre 2 pontos sempre passam por um ponto de controle central.
% Esse ponto em comum analisa a mensagem e envia para o destino

% Usam mensagem do tipo rest, query evento.



% Em 2015, Spencer Lewson descreveu em sua tese de mestrado um protocolo capaz de realizar a comunicação entre nodes em uma névoa. 
% Esse protocolo 

% No que diz respeito a protocolo de comunicação 

% Spencer Lewson escreveu em sua tese de mestrado entitulada "Fog Protocol and FogKit: A JSON-Based Protocol and Framework for Communica- tion Between Bluetooth-Enabled Wearable Internet of Things Devices"

% Podemos tomar como ponto de partida o a tese de mestrado de Spencer Lewson 

% Segundo abc..


