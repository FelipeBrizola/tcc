%----------------------------------------------------------------------------------
% Exemplo do uso da classe tcc.cls. Veja o arquivo .cls
% para mais detalhes e instruções.
%----------------------------------------------------------------------------------
\chapter{\label{chap:intro}Introdução}

%%%%%%%%% TEM QUE ESTAR EM ORDEM ALFABETICA %%%%%%%%%%

%\sigla{CC}{Cloud Computing}

\sigla{BGP}{Border Gateway Protocol}
\sigla{CoAP}{Constrained Application Protocol}
\sigla{HTTP}{Hypertext Transfer Protocol}
\sigla{IP}{Internet Protocol}
\sigla{LE}{Low Energy}
\sigla{NIST}{National Institute of Standards and Technology}
\sigla{TCC}{Trabalho de Conclusão de Curso}
\sigla{TCP}{Transmission Control Protocol}
\sigla{RFC}{Request for Comments}
\sigla{UDP}{User Datagram Protocol}

% Comando para inserir abreviaturas.
%
% \abrev{Abrev}{Abreviatura}
% \abrev{Inform}{Informática}
%
% Comando para inserir símbolos. Estes irão aparecer em ordem
% de ocorrência, já que o número da página está presente na lista
% de símbolos.
% \simbolo{Hz}{Hertz}
% \simbolo{$\pi$}{Constante com valor aproximado de $3.1415926$}%
%
% bom site sobre BRTs
% http://www.brtdata.org/location/latin_america/mexico/mexico_city
%

\section{Contextualização}

A computação em nuvem tem sido amplamente adotada nos últimos anos por usuários finais e empresas de vários segmentos e portes.
As facilidades proporcionadas por este modelo de computação faz com que exista tal preferência, pois segundo a definição adotada
pelo NIST \cite{Mell:2011}: "computação em nuvem é um modelo que permite acesso a um conjunto compartilhado de recursos computacionais configuráveis (por exemplo, redes, servidores, armazenamento, aplicativos e serviços) que podem ser provisionados e liberados  com um pequeno esforço de gerenciamento ou interação com provedor de acesso".
Toda essa flexibilidade pode justificar o aumento no emprego desse modelo de computação.

Como a computação em nuvem não é Panacéia\footnote{Mecanismos ou práticas que, hipoteticamente, são capazes de solucionar os problemas e/ou dificuldades \cite{definition:panaceia}.}, podemos utilizar a definição de  Bonomi, Milito, Zhu e Addepalli \cite{Bonomi:2012} que diz: "a computação em nuvem libera as empresas e os usuários finais de muitos detalhes de especificações.
Essa facilidade torna-se um problema para aplicações sensíveis à latência, que requerem que nós próximos atendam suas necessidades de forma eficiente". 
Como a interação entre os nós e os servidores na nuvem ocorrem através da internet, a baixa latência torna-se indispensável para aplicações que requerem eficiência na comunicação entre nós (por exemplo robôs, drones, e carros autônomos).
Portanto há uma lacuna entre aplicações que já utilizam modelos de computação em nuvem e aplicações que necessitam de baixa latência de rede e comunicação entre nós próximos, e é nesse hiato que a computação em névoa surge.

A computação em névoa é um assunto relativamente novo e teve sua primeira definição, dada pela Cisco Systems\footnote{Empresa estadunidense líder na fabricação de equipamenteos de rede \cite{ciscoSystems}.} em 2012, como uma extensão do paradigma de computação em nuvem provendo armazenamento, computação e serviços de rede entre dispositivos finais e os servidores na nuvem \cite{DBLP:journals/corr/RomanLM16}. 

Atualmente a computação em névoa tornou-se um paradigma próprio e não mais uma mera extensão da computação em nuvem.
Esse paradigma criou o conceito de \textit{fog nodes}, que abrangem desde dispositivos finais com baixa capacidade computacional até servidores poderosos na nuvem.
Assim, os \textit{fog nodes} passam a fazer parte da implementação dos serviços em nuvem.
O que torna a computação em névoa interessante é a capacidade dessa variedade de dispositivos cooperarem uns com os outros de forma distribuída.

Em uma abordagem \textit{bottom-up}, podemos descrever a arquiterura da computação em névoa como um conjunto de \textit{edge devices}\footnote{Dispositivo que controla o fluxo de dados no limite entre duas redes \cite{edgeDevices}.} que se comunicam com os \textit{fog nodes}, e esses com servidores centrais.
Entretanto, a comunicação entre os \textit{fog nodes} e os servidores centrais nao é essencial para a execução dos serviços em névoa \cite{DBLP:journals/corr/RomanLM16}.


\section{Motivação e justificativa}

Serão sete os desafios que a computação em névoa deverá enfrentar para tornar-se realidade, segundo Vaquero e Rodero-Merino \cite{Vaquero:2014}.

Os problemas referentes à padronização, descoberta e sincronização são os desafios a seres explorados neste trabalho, uma vez que atualmente não existem mecanismos
no qual um membro da rede, seja ele uma \textit{raspberry-pi} gerenciando sensores ou um computador, mapeie os recursos disponíveis e divulgue os seus na rede.

\section{Objetivo}

O objetivo principal deste trabalho de conclusão é construir um protocolo de rede que seja capaz de: descobrir, sincronizar e utilizar recursos de nodos em uma rede sob computação em névoa.

% O interfaceamento entre os recursos vinculados aos \textit{fog nodes} serão apresentados em forma de simulação, ou seja,
% os \textit{edge devices} não estarão atrelados aos \textit{fog nodes} de fato.







 
 
 
 
 
 
 
 










