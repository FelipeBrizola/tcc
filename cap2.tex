\chapter{\label{chap:chap2} Fundamentação Teórica}

Este capítulo apresentará uma breve descrição de alguns trabalhos relacionados a área de computação em névoa, bem como protocolos de comunicação que servirão de apoio para este TCC.


\section{Trabalhos relacionados}


O protocolo BGP está situado na quinta camada, a camada de aplicação, do modelo de referência TCP/IP \cite{tanenbaum2011redes}.
Abaixo, de forma sucinta, elencaremos algumas funcionalidades básicas do protocolo que embasarão o restante deste trabalho.

\begin{itemize}
    \item A responsabilidade deste protocolo é manter a troca de informações sobre roteamentos entre sistemas autônomos \cite{Rekhter:1995}.
    \item O roteador ao entrar na rede pela primeira vez deve-se conectar ao seu vizinho. Após a conexão estabelecida, os roteadores compartilham entre sí suas tabelas de roteamento \cite{Rekhter:1995}.
    \item Posteriormente, as atualizações nas tabelas dos roteadores dão-se de forma incremental à medida que as mudanças na rotas ocorrem \cite{Rekhter:1995}.
    \item Mensagens de \textit{keep alive} são trocadas periodicamente a fim de garantir conectividade entre os roteadores \cite{Rekhter:1995}.
\end{itemize}


Spencer Lewson implementou um protocolo em nível de aplicação \cite{tanenbaum2011redes} capaz de realizar a comunicação entre nodos sob computação em névoa.
A especificação do protocolo e um \textit{middleware} capaz de realizar o gerenciamento dos recursos dos dispositivos são os principais componentes deste trabalho \cite{Spencer:2015}.

Sua implementação requer que haja um ponto central de comunicação entre os nodos, uma vez que a conectividade entre eles ocorre via \textit{Bluetooth LE}.
A existência desse ponto justifica-se pelas regras de implementação do \textit{Bluetooth LE}, na qual descreve dispositivos de duas naturezas: centrais e periféricos.
Dispositivos centrais são responsáveis por descobrir dispositivos periféricos que estão interessados em criar conexão.
Portanto, a característica do \textit{Bluetooth LE} faz com que a topologia de rede e a arquitetura do projeto não seja distribuída \cite{Spencer:2015}.








